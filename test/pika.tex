\section{Comparison of Enforcement Approaches for Property Combinations}

This document analyzes three different enforcement approaches:
\begin{enumerate}
    \item \textbf{Serial Composition:} Properties are enforced sequentially
    \item \textbf{Parallel Composition with LCS:} Properties are enforced independently and results are combined using longest common subsequence
    \item \textbf{Monolithic Enforcement:} Properties are combined into a single automaton using product construction
\end{enumerate}

\section{Safety-Safety Properties}

\subsection{Serial Composition: Unbounded Buffer}
\begin{tikzpicture}[node distance=2.5cm]
    % Blocks
    \node (input) at (0,0) {$\sigma \in \Sigma^*$};
    \node[block] (Ephi1) [right of=input] {$E_{\varphi_1}$};
    \node[block] (Ephi2) [right of=Ephi1] {$E_{\varphi_2}$};
    \node (output) [right of=Ephi2, xshift=1.5cm] {$E_{\varphi_2}(E_{\varphi_1}(\sigma)) \in \varphi_1 \cap \varphi_2$};

    % Labels
    \node[above of=Ephi1, yshift=-1.2cm] {$E_{\varphi_1}(\sigma) \in \varphi_1$};
    \node[above of=Ephi2, yshift=-1.2cm] {$E_{\varphi_2} \implies E_{\varphi_2}$};

    % Arrows
    \draw[arrow] (input) -- (Ephi1);
    \draw[arrow] (Ephi1) -- (Ephi2);
    \draw[arrow] (Ephi2) -- (output);
\end{tikzpicture}

\paragraph{} 
The first enforcer $E_{\varphi_1}$ guarantees that the intermediate word $\epsilon_1$ satisfies $\varphi_1$. The second enforcer $E_{\varphi_2}$ guarantees that the final word $\epsilon_2$ satisfies $\varphi_2$. As both safety enforcers suppress unfavorable events, the output word $\epsilon_2 \in \varphi_1 \cap \varphi_2$.

\subsection{Serial Composition: Bounded Buffer}
\paragraph{}
For safety properties with bounded buffers, the results are identical to the unbounded case. This is because unfavorable events are suppressed immediately, and all states except trap states are accepting.

\subsection{Parallel Composition with LCS}
\begin{tikzpicture}[node distance=2.5cm]
    % Blocks
    \node (input) at (0,0) {$\sigma \in \Sigma^*$};
    \node[block] (Ephi1) [above right of=input, yshift=0.5cm] {$E_{\varphi_1}$};
    \node[block] (Ephi2) [below right of=input, yshift=-0.5cm] {$E_{\varphi_2}$};
    \node[block] (LCS) [right of=Ephi1, xshift=1.5cm, yshift=-1.25cm] {LCS};
    \node (output) [right of=LCS, xshift=1cm] {$\text{LCS}(E_{\varphi_1}(\sigma), E_{\varphi_2}(\sigma)) \in \varphi_1 \cap \varphi_2$};

    % Arrows
    \draw[arrow] (input) -- (Ephi1);
    \draw[arrow] (input) -- (Ephi2);
    \draw[arrow] (Ephi1) -- (LCS);
    \draw[arrow] (Ephi2) -- (LCS);
    \draw[arrow] (LCS) -- (output);
\end{tikzpicture}

\paragraph{} 
For safety properties, the parallel LCS approach achieves the same result as serial composition. Each safety enforcer suppresses undesired events, and the LCS preserves only events accepted by both enforcers, yielding a sequence that satisfies $\varphi_1 \cap \varphi_2$.

\paragraph{}
This equivalence holds for both bounded and unbounded buffers because safety properties make immediate suppression decisions without requiring buffering.

\subsection{Monolithic Enforcement}
\begin{tikzpicture}[node distance=2.5cm]
    % Blocks
    \node (input) at (0,0) {$\sigma \in \Sigma^*$};
    \node[block] (Prod) [right of=input, xshift=1cm] {$E_{\varphi_1 \cap \varphi_2}$};
    \node (output) [right of=Prod, xshift=2cm] {$E_{\varphi_1 \cap \varphi_2}(\sigma) \in \varphi_1 \cap \varphi_2$};

    % Arrows
    \draw[arrow] (input) -- (Prod);
    \draw[arrow] (Prod) -- (output);
\end{tikzpicture}

\paragraph{}
The monolithic enforcer directly implements the product construction of both safety automata. For safety properties, all three approaches yield equivalent results with both bounded and unbounded buffers.

\paragraph{}
This is because safety properties only require the suppression of unfavorable events, which all three approaches accomplish effectively.

\section{Co-Safety - Co-Safety Properties}

\subsection{Serial Composition}
\paragraph{}
For an input sequence to satisfy a co-safety property, it must contain a valid prefix. Serial composition works conditionally:
\begin{itemize}
    \item If both enforcers have prefixes such that one valid prefix is a prefix of the other, then the output satisfies both properties
    \item If the prefixes are incompatible, the output may not satisfy the intersection
\end{itemize}

\subsection{Serial Composition: Bounded Buffer}
\paragraph{}
With a bounded buffer, serial composition works correctly only if:
\[
\text{Buffer size} \geq |\text{longer valid prefix}|
\]

\subsection{Parallel Composition with LCS}
\begin{tikzpicture}[node distance=2.5cm]
    % Blocks
    \node (input) at (0,0) {$\sigma \in \Sigma^*$};
    \node[block] (Ephi1) [above right of=input, yshift=0.5cm] {$E_{\varphi_1}$};
    \node[block] (Ephi2) [below right of=input, yshift=-0.5cm] {$E_{\varphi_2}$};
    \node[block] (LCS) [right of=Ephi1, xshift=1.5cm, yshift=-1.25cm] {LCS};
    \node (output) [right of=LCS, xshift=1cm] {$\text{LCS}(E_{\varphi_1}(\sigma), E_{\varphi_2}(\sigma))$};

    % Arrows
    \draw[arrow] (input) -- (Ephi1);
    \draw[arrow] (input) -- (Ephi2);
    \draw[arrow] (Ephi1) -- (LCS);
    \draw[arrow] (Ephi2) -- (LCS);
    \draw[arrow] (LCS) -- (output);
\end{tikzpicture}

\paragraph{}
The parallel LCS approach may fail for co-safety properties because the LCS operation can remove events crucial to satisfying valid prefix requirements. Consider:
\begin{itemize}
    \item Co-safety property 1: Must see pattern "abc"
    \item Co-safety property 2: Must see pattern "adc"
\end{itemize}

For input "abdc":
\begin{align*}
    E_{1}(\text{"abdc"}) &= \text{"abc"} \\
    E_{2}(\text{"abdc"}) &= \text{"adc"} \\
    \text{LCS}(E_{1}, E_{2}) &= \text{"ac"}
\end{align*}

The result "ac" satisfies neither property.

\subsection{Monolithic Enforcement}
\begin{tikzpicture}[node distance=2.5cm]
    % Blocks
    \node (input) at (0,0) {$\sigma \in \Sigma^*$};
    \node[block] (Prod) [right of=input, xshift=1cm] {$E_{\varphi_1 \cap \varphi_2}$};
    \node (output) [right of=Prod, xshift=2cm] {$E_{\varphi_1 \cap \varphi_2}(\sigma)$};

    % Arrows
    \draw[arrow] (input) -- (Prod);
    \draw[arrow] (Prod) -- (output);
\end{tikzpicture}

\paragraph{}
The monolithic enforcer provides optimal results for co-safety properties. By constructing the product automaton, it can precisely identify sequences that satisfy both valid prefix requirements simultaneously. This approach guarantees the output belongs to $\varphi_1 \cap \varphi_2$ when possible, or produces an empty output when no valid solution exists.

\paragraph{}
For bounded buffers, the monolithic enforcer requires a buffer size at least equal to the length of the valid prefix in the product automaton.

\section{Co-Safety - Safety Properties}

\subsection{Serial Composition (Co-Safety followed by Safety)}
\paragraph{}
This composition can fail because the safety enforcer might suppress events that are part of the valid prefix required by the co-safety property.

\paragraph{}
\textbf{Counter-Example:}
\begin{itemize}
    \item Safety property: After a \textit{b} occurs, it is forbidden to have a \textit{c}.
    \item Co-safety property: First 3 actions must be \textit{a, b, c}.
\end{itemize}

For input "abc":
\begin{align*}
    E_{\text{CS}}(\text{"abc"}) &= \text{"abc"} \\
    E_{\text{S}}(E_{\text{CS}}(\text{"abc"})) &= \text{"ab"}
\end{align*}

The output "ab" violates the co-safety property that requires "abc".

\subsection{Serial Composition: Bounded Buffer}
\paragraph{}
The issues with serial composition persist with bounded buffers. If the co-safety enforcer has buffer size sufficient to output its valid prefix, the safety enforcer may still suppress critical events from that prefix.

\subsection{Parallel Composition with LCS}
\begin{tikzpicture}[node distance=2.5cm]
    % Blocks
    \node (input) at (0,0) {$\sigma \in \Sigma^*$};
    \node[block] (ECS) [above right of=input, yshift=0.5cm] {$E_{\varphi_\text{CS}}$};
    \node[block] (ES) [below right of=input, yshift=-0.5cm] {$E_{\varphi_\text{S}}$};
    \node[block] (LCS) [right of=ECS, xshift=1.5cm, yshift=-1.25cm] {LCS};
    \node (output) [right of=LCS, xshift=1cm] {$\text{LCS}(E_{\varphi_\text{CS}}(\sigma), E_{\varphi_\text{S}}(\sigma))$};

    % Arrows
    \draw[arrow] (input) -- (ECS);
    \draw[arrow] (input) -- (ES);
    \draw[arrow] (ECS) -- (LCS);
    \draw[arrow] (ES) -- (LCS);
    \draw[arrow] (LCS) -- (output);
\end{tikzpicture}

\paragraph{}
Using the same example:
\begin{align*}
    E_{\text{CS}}(\text{"abc"}) &= \text{"abc"} \\
    E_{\text{S}}(\text{"abc"}) &= \text{"ab"} \\
    \text{LCS}(E_{\text{CS}}, E_{\text{S}}) &= \text{"ab"}
\end{align*}

The LCS result "ab" violates the co-safety property requiring "abc". The parallel LCS approach faces similar limitations as serial composition when the properties have conflicting requirements.

\subsection{Monolithic Enforcement}
\begin{tikzpicture}[node distance=2.5cm]
    % Blocks
    \node (input) at (0,0) {$\sigma \in \Sigma^*$};
    \node[block] (Prod) [right of=input, xshift=1cm] {$E_{\varphi_\text{CS} \cap \varphi_\text{S}}$};
    \node (output) [right of=Prod, xshift=2cm] {$E_{\varphi_\text{CS} \cap \varphi_\text{S}}(\sigma)$};

    % Arrows
    \draw[arrow] (input) -- (Prod);
    \draw[arrow] (Prod) -- (output);
\end{tikzpicture}

\paragraph{}
For incompatible safety and co-safety properties, the monolithic enforcer correctly identifies that no valid output exists (i.e., $\varphi_\text{S} \cap \varphi_\text{CS} = \emptyset$). When the properties are compatible, it produces the optimal solution that satisfies both properties simultaneously.

\paragraph{}
This approach correctly handles the "abc" example by recognizing that the intersection of the two properties is empty and producing no output.

\section{Safety - Co-Safety Properties}

\subsection{Serial Composition (Safety followed by Co-Safety)}
\paragraph{}
This approach works correctly because enforcing the safety property first removes undesired events, and then the co-safety enforcer ensures the required valid prefix is present. The output satisfies both properties.

\subsection{Serial Composition: Bounded Buffer}
\paragraph{}
With a bounded buffer, this approach works correctly only if:
\[
\text{Buffer size} \geq |\text{valid prefix of the co-safety property}|
\]

\subsection{Parallel Composition with LCS}
\begin{tikzpicture}[node distance=2.5cm]
    % Blocks
    \node (input) at (0,0) {$\sigma \in \Sigma^*$};
    \node[block] (ES) [above right of=input, yshift=0.5cm] {$E_{\varphi_\text{S}}$};
    \node[block] (ECS) [below right of=input, yshift=-0.5cm] {$E_{\varphi_\text{CS}}$};
    \node[block] (LCS) [right of=ES, xshift=1.5cm, yshift=-1.25cm] {LCS};
    \node (output) [right of=LCS, xshift=1cm] {$\text{LCS}(E_{\varphi_\text{S}}(\sigma), E_{\varphi_\text{CS}}(\sigma))$};

    % Arrows
    \draw[arrow] (input) -- (ES);
    \draw[arrow] (input) -- (ECS);
    \draw[arrow] (ES) -- (LCS);
    \draw[arrow] (ECS) -- (LCS);
    \draw[arrow] (LCS) -- (output);
\end{tikzpicture}

\paragraph{}
Unlike the Co-Safety followed by Safety case, the parallel LCS approach can work correctly for compatible properties. However, it may still fail if the safety property suppresses events that are critical to the co-safety property's valid prefix.

\paragraph{}
For compatible properties, the parallel LCS approach produces results equivalent to the serial composition of Safety followed by Co-Safety.

\subsection{Monolithic Enforcement}
\begin{tikzpicture}[node distance=2.5cm]
    % Blocks
    \node (input) at (0,0) {$\sigma \in \Sigma^*$};
    \node[block] (Prod) [right of=input, xshift=1cm] {$E_{\varphi_\text{S} \cap \varphi_\text{CS}}$};
    \node (output) [right of=Prod, xshift=2cm] {$E_{\varphi_\text{S} \cap \varphi_\text{CS}}(\sigma)$};

    % Arrows
    \draw[arrow] (input) -- (Prod);
    \draw[arrow] (Prod) -- (output);
\end{tikzpicture}

\paragraph{}
The monolithic enforcer provides optimal enforcement by directly representing the intersection of both properties. It produces the same results as the serial composition of Safety followed by Co-Safety for compatible properties, and correctly identifies when no valid output exists for incompatible properties.

\section{Regular - Regular Properties}

\subsection{Serial Composition}
\paragraph{}
Serial composition does not guarantee the preservation of both regular properties.

\paragraph{}
\textbf{Counter-Example:}
\begin{itemize}
    \item RE1: Action \textit{a} followed by \textit{b} or \textit{c} should alternate starting with \textit{a}.
    \item RE2: The first action should be \textit{a}, immediately followed by \textit{b}, and immediately followed by another \textit{a}. This sequence can be repeated again after a \textit{c}.
\end{itemize}

For input "abac", both serial compositions fail:
\begin{align*}
    E_{\text{R2}}(E_{\text{R1}}(\text{"abac"})) &= E_{\text{R2}}(\text{"abac"}) = \text{"aba"} \\
    E_{\text{R1}}(E_{\text{R2}}(\text{"abac"})) &= E_{\text{R1}}(\text{"aba"}) = \text{"ab"}
\end{align*}

Neither result satisfies both R1 and R2.

\subsection{Serial Composition: Bounded Buffer}
\paragraph{}
The failures of serial composition with regular properties persist with bounded buffers. The issue is fundamental to the enforcement mechanism, not the buffer size.

\subsection{Parallel Composition with LCS}
\begin{tikzpicture}[node distance=2.5cm]
    % Blocks
    \node (input) at (0,0) {$\sigma \in \Sigma^*$};
    \node[block] (ER1) [above right of=input, yshift=0.5cm] {$E_{\text{R1}}$};
    \node[block] (ER2) [below right of=input, yshift=-0.5cm] {$E_{\text{R2}}$};
    \node[block] (LCS) [right of=ER1, xshift=1.5cm, yshift=-1.25cm] {LCS};
    \node (output) [right of=LCS, xshift=1cm] {$\text{LCS}(E_{\text{R1}}(\sigma), E_{\text{R2}}(\sigma))$};

    % Arrows
    \draw[arrow] (input) -- (ER1);
    \draw[arrow] (input) -- (ER2);
    \draw[arrow] (ER1) -- (LCS);
    \draw[arrow] (ER2) -- (LCS);
    \draw[arrow] (LCS) -- (output);
\end{tikzpicture}

\paragraph{}
Using the same example:
\begin{align*}
    E_{\text{R1}}(\text{"abac"}) &= \text{"abac"} \\
    E_{\text{R2}}(\text{"abac"}) &= \text{"aba"} \\
    \text{LCS}(E_{\text{R1}}, E_{\text{R2}}) &= \text{"aba"}
\end{align*}

The LCS result "aba" satisfies R2 but not R1, so it doesn't satisfy the intersection. Parallel LCS composition also fails to guarantee enforcement of both regular properties simultaneously.

\subsection{Monolithic Enforcement}
\begin{tikzpicture}[node distance=2.5cm]
    % Blocks
    \node (input) at (0,0) {$\sigma \in \Sigma^*$};
    \node[block] (Prod) [right of=input, xshift=1cm] {$E_{\text{R1} \cap \text{R2}}$};
    \node (output) [right of=Prod, xshift=2cm] {$E_{\text{R1} \cap \text{R2}}(\sigma)$};

    % Arrows
    \draw[arrow] (input) -- (Prod);
    \draw[arrow] (Prod) -- (output);
\end{tikzpicture}

\paragraph{}
The monolithic enforcer correctly identifies when there is no output that can satisfy both properties simultaneously.
\end{document}


